%!TEX root = paper.tex
\begin{abstract}
Multiple outages in major mobile networks have been reported in the recent past. In fixed and datacenter networks such capacity problems are solved by scaling out, i.e. purchasing additional hardware. In mobile networks this is not as easily possible as network components are usually sold as sealed middleboxes. With the advances in server performance and SDN it has been suggested to virtualize these boxes. This also opens up opportunities to dimension according to current load and save energy by switching off parts of the infrastructure.

Such suggestions immediately raise questions on the cost of virtualization. To answer this, we introduce models for both a traditional as well as a virtualized GGSN. In addition, we provide distributions for the load experienced at the GGSN based on network measurements. With this at hand, we study the influence of different dimensioning parameters on important performance metrics, with special consideration for the impact of provisioning new instances for the virtual GGSN.
\end{abstract}
