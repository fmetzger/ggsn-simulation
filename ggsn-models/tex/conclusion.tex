%!TEX root = paper.tex
\section{Conclusion}
\label{sec:conclusion}
In this paper we investigated models and trade-offs for virtualizing components of the mobile core network.
We first discussed a novel approach to mobile core network load modeling based on the control plane load at the \gls{GGSN}. The non-stationary Erlang loss model $M(t)/G/c_c/0$ is based on the currently implemented state of the network architecture and backed by an evaluation of actual data. This can serve as a baseline reference to plan and dimension mobile network accordingly, not just based on expected user traffic as traditionally.
To improve scaling in the future, we proposed a new and virtualizable approach for \glspl{GGSN}.
We presented random variables to model load in a \gls{GGSN} based on measurement data from the network of a nation-wide mobile service provider and made them available for reuse.
Finally, we evaluate the model using a queuing simulation. We have shown, that the system's blocking probability is roughly equal to the single-server model but in addition achieves large efficiency gains, even when subjected to rudimentary provisioning conditions and long boot times.
The model also has the ability to very easily scale out one's infrastructure by simply adding more small servers, reducing operational overhead.
Implementing this model in an actual network might need considerable future effort and adaptation of existing infrastructure, protocols and standards. But if done correctly it could lead to new GGSN-as-a-Service business models, removing the need to provide and operate large amounts of infrastructure for rare cases of peak load. 

%In the future we would like to deepen our modeling efforts to provide more dimensioning options for a core network.
% Also, we want to further investigate the correlation of user traffic and signaling and take a look at the implications specific traffic types bring for the core network. 
